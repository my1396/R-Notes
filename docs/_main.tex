% Options for packages loaded elsewhere
\PassOptionsToPackage{unicode}{hyperref}
\PassOptionsToPackage{hyphens}{url}
%
\documentclass[
]{book}
\usepackage{amsmath,amssymb}
\usepackage{iftex}
\ifPDFTeX
  \usepackage[T1]{fontenc}
  \usepackage[utf8]{inputenc}
  \usepackage{textcomp} % provide euro and other symbols
\else % if luatex or xetex
  \usepackage{unicode-math} % this also loads fontspec
  \defaultfontfeatures{Scale=MatchLowercase}
  \defaultfontfeatures[\rmfamily]{Ligatures=TeX,Scale=1}
\fi
\usepackage{lmodern}
\ifPDFTeX\else
  % xetex/luatex font selection
\fi
% Use upquote if available, for straight quotes in verbatim environments
\IfFileExists{upquote.sty}{\usepackage{upquote}}{}
\IfFileExists{microtype.sty}{% use microtype if available
  \usepackage[]{microtype}
  \UseMicrotypeSet[protrusion]{basicmath} % disable protrusion for tt fonts
}{}
\makeatletter
\@ifundefined{KOMAClassName}{% if non-KOMA class
  \IfFileExists{parskip.sty}{%
    \usepackage{parskip}
  }{% else
    \setlength{\parindent}{0pt}
    \setlength{\parskip}{6pt plus 2pt minus 1pt}}
}{% if KOMA class
  \KOMAoptions{parskip=half}}
\makeatother
\usepackage{xcolor}
\usepackage{color}
\usepackage{fancyvrb}
\newcommand{\VerbBar}{|}
\newcommand{\VERB}{\Verb[commandchars=\\\{\}]}
\DefineVerbatimEnvironment{Highlighting}{Verbatim}{commandchars=\\\{\}}
% Add ',fontsize=\small' for more characters per line
\usepackage{framed}
\definecolor{shadecolor}{RGB}{248,248,248}
\newenvironment{Shaded}{\begin{snugshade}}{\end{snugshade}}
\newcommand{\AlertTok}[1]{\textcolor[rgb]{0.94,0.16,0.16}{#1}}
\newcommand{\AnnotationTok}[1]{\textcolor[rgb]{0.56,0.35,0.01}{\textbf{\textit{#1}}}}
\newcommand{\AttributeTok}[1]{\textcolor[rgb]{0.13,0.29,0.53}{#1}}
\newcommand{\BaseNTok}[1]{\textcolor[rgb]{0.00,0.00,0.81}{#1}}
\newcommand{\BuiltInTok}[1]{#1}
\newcommand{\CharTok}[1]{\textcolor[rgb]{0.31,0.60,0.02}{#1}}
\newcommand{\CommentTok}[1]{\textcolor[rgb]{0.56,0.35,0.01}{\textit{#1}}}
\newcommand{\CommentVarTok}[1]{\textcolor[rgb]{0.56,0.35,0.01}{\textbf{\textit{#1}}}}
\newcommand{\ConstantTok}[1]{\textcolor[rgb]{0.56,0.35,0.01}{#1}}
\newcommand{\ControlFlowTok}[1]{\textcolor[rgb]{0.13,0.29,0.53}{\textbf{#1}}}
\newcommand{\DataTypeTok}[1]{\textcolor[rgb]{0.13,0.29,0.53}{#1}}
\newcommand{\DecValTok}[1]{\textcolor[rgb]{0.00,0.00,0.81}{#1}}
\newcommand{\DocumentationTok}[1]{\textcolor[rgb]{0.56,0.35,0.01}{\textbf{\textit{#1}}}}
\newcommand{\ErrorTok}[1]{\textcolor[rgb]{0.64,0.00,0.00}{\textbf{#1}}}
\newcommand{\ExtensionTok}[1]{#1}
\newcommand{\FloatTok}[1]{\textcolor[rgb]{0.00,0.00,0.81}{#1}}
\newcommand{\FunctionTok}[1]{\textcolor[rgb]{0.13,0.29,0.53}{\textbf{#1}}}
\newcommand{\ImportTok}[1]{#1}
\newcommand{\InformationTok}[1]{\textcolor[rgb]{0.56,0.35,0.01}{\textbf{\textit{#1}}}}
\newcommand{\KeywordTok}[1]{\textcolor[rgb]{0.13,0.29,0.53}{\textbf{#1}}}
\newcommand{\NormalTok}[1]{#1}
\newcommand{\OperatorTok}[1]{\textcolor[rgb]{0.81,0.36,0.00}{\textbf{#1}}}
\newcommand{\OtherTok}[1]{\textcolor[rgb]{0.56,0.35,0.01}{#1}}
\newcommand{\PreprocessorTok}[1]{\textcolor[rgb]{0.56,0.35,0.01}{\textit{#1}}}
\newcommand{\RegionMarkerTok}[1]{#1}
\newcommand{\SpecialCharTok}[1]{\textcolor[rgb]{0.81,0.36,0.00}{\textbf{#1}}}
\newcommand{\SpecialStringTok}[1]{\textcolor[rgb]{0.31,0.60,0.02}{#1}}
\newcommand{\StringTok}[1]{\textcolor[rgb]{0.31,0.60,0.02}{#1}}
\newcommand{\VariableTok}[1]{\textcolor[rgb]{0.00,0.00,0.00}{#1}}
\newcommand{\VerbatimStringTok}[1]{\textcolor[rgb]{0.31,0.60,0.02}{#1}}
\newcommand{\WarningTok}[1]{\textcolor[rgb]{0.56,0.35,0.01}{\textbf{\textit{#1}}}}
\usepackage{longtable,booktabs,array}
\usepackage{calc} % for calculating minipage widths
% Correct order of tables after \paragraph or \subparagraph
\usepackage{etoolbox}
\makeatletter
\patchcmd\longtable{\par}{\if@noskipsec\mbox{}\fi\par}{}{}
\makeatother
% Allow footnotes in longtable head/foot
\IfFileExists{footnotehyper.sty}{\usepackage{footnotehyper}}{\usepackage{footnote}}
\makesavenoteenv{longtable}
\usepackage{graphicx}
\makeatletter
\newsavebox\pandoc@box
\newcommand*\pandocbounded[1]{% scales image to fit in text height/width
  \sbox\pandoc@box{#1}%
  \Gscale@div\@tempa{\textheight}{\dimexpr\ht\pandoc@box+\dp\pandoc@box\relax}%
  \Gscale@div\@tempb{\linewidth}{\wd\pandoc@box}%
  \ifdim\@tempb\p@<\@tempa\p@\let\@tempa\@tempb\fi% select the smaller of both
  \ifdim\@tempa\p@<\p@\scalebox{\@tempa}{\usebox\pandoc@box}%
  \else\usebox{\pandoc@box}%
  \fi%
}
% Set default figure placement to htbp
\def\fps@figure{htbp}
\makeatother
\setlength{\emergencystretch}{3em} % prevent overfull lines
\providecommand{\tightlist}{%
  \setlength{\itemsep}{0pt}\setlength{\parskip}{0pt}}
\setcounter{secnumdepth}{5}
\usepackage{booktabs}
\usepackage[]{natbib}
\bibliographystyle{plainnat}
\usepackage{bookmark}
\IfFileExists{xurl.sty}{\usepackage{xurl}}{} % add URL line breaks if available
\urlstyle{same}
\hypersetup{
  pdftitle={A Minimal Book Example},
  pdfauthor={John Doe},
  hidelinks,
  pdfcreator={LaTeX via pandoc}}

\title{A Minimal Book Example}
\author{John Doe}
\date{2025-03-07}

\begin{document}
\maketitle

{
\setcounter{tocdepth}{1}
\tableofcontents
}
\chapter{About}\label{about}

This is a \emph{sample} book written in \textbf{Markdown}. You can use anything that Pandoc's Markdown supports; for example, a math equation \(a^2 + b^2 = c^2\).

\section{Usage}\label{usage}

Each \textbf{bookdown} chapter is an .Rmd file, and each .Rmd file can contain one (and only one) chapter. A chapter \emph{must} start with a first-level heading: \texttt{\#\ A\ good\ chapter}, and can contain one (and only one) first-level heading.

Use second-level and higher headings within chapters like: \texttt{\#\#\ A\ short\ section} or \texttt{\#\#\#\ An\ even\ shorter\ section}.

The \texttt{index.Rmd} file is required, and is also your first book chapter. It will be the homepage when you render the book.

\section{Render book}\label{render-book}

You can render the HTML version of this example book without changing anything:

\begin{enumerate}
\def\labelenumi{\arabic{enumi}.}
\item
  Find the \textbf{Build} pane in the RStudio IDE, and
\item
  Click on \textbf{Build Book}, then select your output format, or select ``All formats'' if you'd like to use multiple formats from the same book source files.
\end{enumerate}

Or build the book from the R console:

\begin{Shaded}
\begin{Highlighting}[]
\NormalTok{bookdown}\SpecialCharTok{::}\FunctionTok{render\_book}\NormalTok{()}
\end{Highlighting}
\end{Shaded}

To render this example to PDF as a \texttt{bookdown::pdf\_book}, you'll need to install XeLaTeX. You are recommended to install TinyTeX (which includes XeLaTeX): \url{https://yihui.org/tinytex/}.

\section{Preview book}\label{preview-book}

As you work, you may start a local server to live preview this HTML book. This preview will update as you edit the book when you save individual .Rmd files. You can start the server in a work session by using the RStudio add-in ``Preview book'', or from the R console:

\begin{Shaded}
\begin{Highlighting}[]
\NormalTok{bookdown}\SpecialCharTok{::}\FunctionTok{serve\_book}\NormalTok{()}
\end{Highlighting}
\end{Shaded}

\chapter{Rstudio}\label{rstudio}

\textbf{Rstudio shortcuts}

\begin{longtable}[]{@{}
  >{\raggedright\arraybackslash}p{(\linewidth - 2\tabcolsep) * \real{0.4783}}
  >{\raggedright\arraybackslash}p{(\linewidth - 2\tabcolsep) * \real{0.5217}}@{}}
\toprule\noalign{}
\begin{minipage}[b]{\linewidth}\raggedright
keyboard combination
\end{minipage} & \begin{minipage}[b]{\linewidth}\raggedright
function
\end{minipage} \\
\midrule\noalign{}
\endhead
\bottomrule\noalign{}
\endlastfoot
opt + \_ & insert assignment operator \texttt{\textless{}-} \\
ESC or ctrl + C & exit \texttt{+} prompt \\
ctrl + shift + m & add pipe operator ``\%\textgreater\%'' \\
{ctrl + \texttt{{[}}/\texttt{{]}} } & indent or unindent \\
cmd + D & delete one row \\
cmd + 1 & move cursor to console window \\
cmd + 2 & move cursor to editor window \\
ctrl + shift + S & add 80 hyphens \texttt{-\/-\/-} to signal a new chapter (Addin) \\
ctrl + shift + = & add 80 equals \texttt{===} to signal a new Chapter (Addin) \\
shift + cmd +N & new R script \\
cmd + \(\uparrow\) / \(\downarrow\) & in console, get a list of command history \\
shift + \(\uparrow\) / \(\downarrow\) & select one line up/down \\
fn + F2 & \texttt{view()} an object, don't select the object \\
cmd + shift + 1 & activate X11() window \\
{ctrl (+ shift) + tab} & next (last) tab in scriptor (this applies to all apps); hit ctrl first, then shift if necessary, last tab \\
\end{longtable}

{\textbf{Source}}

\begin{longtable}[]{@{}
  >{\raggedright\arraybackslash}p{(\linewidth - 2\tabcolsep) * \real{0.2817}}
  >{\raggedright\arraybackslash}p{(\linewidth - 2\tabcolsep) * \real{0.7183}}@{}}
\toprule\noalign{}
\begin{minipage}[b]{\linewidth}\raggedright
keyboard combination
\end{minipage} & \begin{minipage}[b]{\linewidth}\raggedright
function
\end{minipage} \\
\midrule\noalign{}
\endhead
\bottomrule\noalign{}
\endlastfoot
cmd + return & Run current line/selection \\
opt + return & Run current line/selection (retain cursor position) \\
\end{longtable}

{\textbf{\texttt{Rmd} related}}

\begin{longtable}[]{@{}
  >{\raggedright\arraybackslash}p{(\linewidth - 2\tabcolsep) * \real{0.2500}}
  >{\raggedright\arraybackslash}p{(\linewidth - 2\tabcolsep) * \real{0.7500}}@{}}
\toprule\noalign{}
\begin{minipage}[b]{\linewidth}\raggedright
keyboard combination
\end{minipage} & \begin{minipage}[b]{\linewidth}\raggedright
function
\end{minipage} \\
\midrule\noalign{}
\endhead
\bottomrule\noalign{}
\endlastfoot
cmd + shift + K & \textbf{Knit} rmd \\
cmd + opt + C & run current code chunk in \texttt{Rmd} \\
cmd + opt + I & insert code chunks in \texttt{Rmd}, i.e., \texttt{\textasciigrave{}\textasciigrave{}\textasciigrave{}\{r\}} and \texttt{\textasciigrave{}\textasciigrave{}\textasciigrave{}} \\
\end{longtable}

Q: How to print output in console rather than inline in Rmd?

A: Choose the gear in the editor toolbar and choose ``Chunk Output in Console''.

{\textbf{Set working directory}}

\begin{Shaded}
\begin{Highlighting}[]
\NormalTok{dir\_folder }\OtherTok{\textless{}{-}} \FunctionTok{dirname}\NormalTok{(rstudioapi}\SpecialCharTok{::}\FunctionTok{getSourceEditorContext}\NormalTok{()}\SpecialCharTok{$}\NormalTok{path) }\CommentTok{\# get the dir name of the current script}
\FunctionTok{setwd}\NormalTok{(dir\_folder) }\CommentTok{\# set as working dir}
\end{Highlighting}
\end{Shaded}

RStudio projects are associated with R working directories. You can create an RStudio project:

\begin{itemize}
\tightlist
\item
  In a brand new directory
\item
  In an existing directory where you already have R code and data
\item
  By cloning a version control (Git or Subversion) repository
\end{itemize}

Why using R projects:

\begin{enumerate}
\def\labelenumi{\arabic{enumi}.}
\tightlist
\item
  I don't need to use \texttt{setwd} at the start of each script, and if I move the base project folder it will still work.
\item
  I have a personal package with a custom project, which creates my folders just the way I like them. This makes it so that the basic locations for data, outputs and analysis is the same across my work.
\end{enumerate}

Double-click on a \texttt{.Rproj} file to open a fresh instance of RStudio, with the working directory and file browser pointed at the project folder.

Q: What is an \textbf{R session}? And when do I use it?

A: Multiple concurrent sessions can be useful when you want to:

\begin{itemize}
\tightlist
\item
  Run multiple analyses in parallel
\item
  Keep multiple sessions open indefinitely
\item
  Participate in one or more \href{https://support.posit.co/hc/en-us/articles/211659737}{shared projects}
\end{itemize}

\textbf{Lauch a new project-less RStudio session}

\begin{Shaded}
\begin{Highlighting}[]
\CommentTok{\# run in console}
\NormalTok{rstudioapi}\SpecialCharTok{::}\FunctionTok{terminalExecute}\NormalTok{(}\StringTok{"open {-}n /Applications/RStudio.app"}\NormalTok{, }\AttributeTok{show =} \ConstantTok{FALSE}\NormalTok{)}
\end{Highlighting}
\end{Shaded}

\texttt{-n} Open a new instance of the application(s) even if one is already running.

\texttt{rstudioapi::terminalExecute(command,\ workingDir\ =\ NULL,\ env\ =\ character(),\ show\ =\ TRUE)} tells R to run the system command in quotes.

\begin{itemize}
\tightlist
\item
  \texttt{command} System command to be invoked, as a character string.
\item
  \texttt{workingDir} Working directory for command
\item
  \texttt{env} Vector of name=value strings to set environment variables
\item
  \texttt{show} If FALSE, terminal won't be brought to front
\end{itemize}

The {\texttt{rstudioapi}} package provides an interface for interacting with the RStudio IDE with R code. Using\texttt{rstudioapi}, you can:

\begin{itemize}
\tightlist
\item
  Examine, manipulate, and save the contents of documents currently open in RStudio,
\item
  Create, open, or re-open RStudio projects,
\item
  Prompt the user with different kinds of dialogs (e.g.~for selecting a file or folder, or requesting a password from the user),
\item
  Interact with RStudio terminals,
\item
  Interact with the R session associated with the current RStudio instance.
\end{itemize}

\begin{center}\rule{0.5\linewidth}{0.5pt}\end{center}

\textbf{Set up Development Tools}

\url{https://cran.r-project.org/bin/macosx/tools/}

\begin{itemize}
\item
  install Xcode command line tools

\begin{Shaded}
\begin{Highlighting}[]
\FunctionTok{sudo}\NormalTok{ xcode{-}select }\AttributeTok{{-}{-}install}
\end{Highlighting}
\end{Shaded}
\item
  install GNU Fortran compiler

  Using \textbf{Apple silicon} (aka arm64, aarch64, M1) Macs Fortran compiler
\item
  Go to \url{https://www.xquartz.org/}, download the .dmg and run the installer.
\item
  Verify that build tools are installed and available by opening an R console and running

\begin{Shaded}
\begin{Highlighting}[]
\FunctionTok{install.packages}\NormalTok{(}\StringTok{"pkgbuild"}\NormalTok{)}
\NormalTok{pkgbuild}\SpecialCharTok{::}\FunctionTok{check\_build\_tools}\NormalTok{()}
\end{Highlighting}
\end{Shaded}
\end{itemize}

\begin{center}\rule{0.5\linewidth}{0.5pt}\end{center}

\textbf{Insert Code Session}

To insert a new code section you can use the \textbf{Code} -\textgreater{} \textbf{Insert Section} command. Alternatively, any comment line which includes at least four trailing dashes (\texttt{-}), equal signs (\texttt{=}), or pound signs (\texttt{\#}) automatically creates a code section.

\textbf{Define your own shortcuts}

\url{https://www.statworx.com/ch/blog/defining-your-own-shortcut-in-rstudio/}

\url{https://www.r-bloggers.com/2020/03/defining-your-own-shortcut-in-rstudio/}

Install the shortcut packages.

Add code session separators, \texttt{-\/-\/-} or \texttt{===}.

\begin{Shaded}
\begin{Highlighting}[]
\FunctionTok{install.packages}\NormalTok{(}
    \CommentTok{\# same path as above}
  \StringTok{"\textasciitilde{}/Downloads/shoRtcut\_0.1.0.tar.gz"}\NormalTok{, }
  \CommentTok{\# indicate it is a local file}
  \AttributeTok{repos =} \ConstantTok{NULL}\NormalTok{)}
\FunctionTok{install.packages}\NormalTok{(}
    \CommentTok{\# same path as above}
  \StringTok{"\textasciitilde{}/Downloads/shoRtcut2\_0.1.0.tar.gz"}\NormalTok{, }
  \CommentTok{\# indicate it is a local file}
  \AttributeTok{repos =} \ConstantTok{NULL}\NormalTok{)}
\end{Highlighting}
\end{Shaded}

Now go to Tools \textgreater{} Modify Keyboard Shortcuts and search for ``dashes''. Here you can define the keyboard combination by clicking inside the empty Shortcut field and pressing the desired key-combination on your keyboard. Click Apply, and that's it!

\begin{center}\rule{0.5\linewidth}{0.5pt}\end{center}

\section{Dark Theme}\label{dark-theme}

\url{https://community.rstudio.com/t/fvaleature-req-word-background-highlight-color-in-find-and-spellcheck/18578/3}

\url{https://rstudio.github.io/rstudio-extensions/rstudio-theme-creation.html}

\url{https://docs.posit.co/ide/user/ide/guide/ui/appearance.html\#creating-custom-themes-for-rstudio}

\texttt{.ace\_marker-layer\ .ace\_selection} Changes the color and style of the highlighting for the currently selected line or block of lines.

\texttt{.ace\_marker-layer\ .ace\_bracket} Changes the color and style of the highlighting on matching brackets.

\textbf{Recommended highlight color}: \texttt{rgba(255,\ 0,\ 0,\ 0.47)}

\texttt{RStudio} editor theme directory on Mac:

right click \texttt{RStudio.app}, ``Show Package Contents'' to navigate to the application folder.

\texttt{/Applications/RStudio.app/Contents/Resources/resources/themes/ambiance.rstheme}

Custom theme (user-defined) folder:

\begin{itemize}
\tightlist
\item
  \texttt{\textasciitilde{}/.config/rstudio/themes/idle\_fingers\_2.rstheme} on mac
\item
  \href{https://github.com/z3tt/viridis-theme/blob/main/viridis.rstheme}{viridis-theme}
\end{itemize}

\begin{Shaded}
\begin{Highlighting}[]
\CommentTok{/* yaml tag */}
\FunctionTok{.ace\_meta.ace\_tag}\NormalTok{ \{}
  \KeywordTok{color}\CharTok{:} \ConstantTok{\#2499DA}\OperatorTok{;}
\NormalTok{\}}
\CommentTok{/* quoted by $...$ and code chunk options */}
\FunctionTok{.ace\_support.ace\_function}\NormalTok{ \{}
  \KeywordTok{color}\CharTok{:} \ConstantTok{\#55C667}\OperatorTok{;}
\NormalTok{\}}
\end{Highlighting}
\end{Shaded}

\begin{center}\rule{0.5\linewidth}{0.5pt}\end{center}

\section{Packages Management}\label{packages-management}

\textbf{Install R pakages from source}

\begin{Shaded}
\begin{Highlighting}[]
\CommentTok{\# From local tarball}
\FunctionTok{install.packages}\NormalTok{(}
  \CommentTok{\# indicate path of the package source file}
  \StringTok{"\textasciitilde{}/Documents/R/UserPackages/shoRtcut2\_0.1.0.tar.gz"}\NormalTok{, }
  \CommentTok{\# indicate it is a local file}
  \AttributeTok{repos =} \ConstantTok{NULL}\NormalTok{)}

\CommentTok{\# From github}
\FunctionTok{install.packages}\NormalTok{(}\StringTok{"Rcpp"}\NormalTok{, }\AttributeTok{repos=}\StringTok{"https://rcppcore.github.io/drat"}\NormalTok{)}
\end{Highlighting}
\end{Shaded}

Check insatlled packages

\begin{Shaded}
\begin{Highlighting}[]
\CommentTok{\# print all installed packages}
\FunctionTok{rownames}\NormalTok{(}\FunctionTok{installed.packages}\NormalTok{())}
\CommentTok{\# check if \textasciigrave{}ggplot2\textasciigrave{} is installed}
\StringTok{"ggplot2"} \SpecialCharTok{\%in\%} \FunctionTok{rownames}\NormalTok{(}\FunctionTok{installed.packages}\NormalTok{())}
\end{Highlighting}
\end{Shaded}

Update packages

\begin{Shaded}
\begin{Highlighting}[]
\FunctionTok{packageVersion}\NormalTok{(}\StringTok{"ggplot2"}\NormalTok{) }\CommentTok{\# check package version}
\FunctionTok{install.packages}\NormalTok{(}\StringTok{"ggplot2"}\NormalTok{) }\CommentTok{\# update one specific package}

\DocumentationTok{\#\# update all installed packages in a stated library location, default to \textasciigrave{}.libPaths()\textasciigrave{}}
\FunctionTok{update.packages}\NormalTok{(}\AttributeTok{lib.loc =} \FunctionTok{.libPaths}\NormalTok{()) }
\end{Highlighting}
\end{Shaded}

Which will ask you for every package if you want to update, to just say yes to everything use \texttt{ask\ =\ FAlSE}.

\begin{Shaded}
\begin{Highlighting}[]
\FunctionTok{update.packages}\NormalTok{(}\AttributeTok{ask =} \ConstantTok{FALSE}\NormalTok{)}
\end{Highlighting}
\end{Shaded}

Unfortunately this {won't} update packages installed by \texttt{devtools::install\_github()}

\textbf{Updating all Packages after {R update}}

R packages are missing after updating. So have to save the installed packages and re-install them after updating.

\begin{Shaded}
\begin{Highlighting}[]
\DocumentationTok{\#\# get packages installed}
\NormalTok{packs }\OtherTok{\textless{}{-}} \FunctionTok{as.data.frame}\NormalTok{(}\FunctionTok{installed.packages}\NormalTok{(}\FunctionTok{.libPaths}\NormalTok{()[}\DecValTok{1}\NormalTok{]), }\AttributeTok{stringsAsFactors =}\NormalTok{ F)}
\CommentTok{\# Save to local}
\NormalTok{f\_name }\OtherTok{\textless{}{-}} \StringTok{"\textasciitilde{}/Documents/R/packages.csv"}
\FunctionTok{rownames}\NormalTok{(packs)}
\FunctionTok{write.csv}\NormalTok{(packs, f\_name, }\AttributeTok{row.names =} \ConstantTok{FALSE}\NormalTok{)}
\NormalTok{packs }\OtherTok{\textless{}{-}} \FunctionTok{read\_csv}\NormalTok{(f\_name)}
\NormalTok{packs}
\DocumentationTok{\#\# Re{-}install packages using install.packages() after updating R}
\FunctionTok{install.packages}\NormalTok{(packs}\SpecialCharTok{$}\NormalTok{Package)}
\end{Highlighting}
\end{Shaded}

R library path \texttt{/Library/Frameworks/R.framework/Versions/4.2-arm64/Resources/library}

\begin{itemize}
\tightlist
\item
  use \texttt{find.package("ggplot2")} to find the location of the source file.
\item
  alternatively, \texttt{.libPaths()}

  \begin{itemize}
  \tightlist
  \item
    returns the directory within which packages are looked for.
  \end{itemize}
\end{itemize}

\texttt{library(package)} returns an error if the package doesn't exist.

\texttt{require(package)} returns \texttt{FALSE} if the package doesn't exist. \texttt{require} is designed for use inside other functions.

\begin{center}\rule{0.5\linewidth}{0.5pt}\end{center}

\subsection*{Put your R package on GitHub}\label{put-your-r-package-on-github}
\addcontentsline{toc}{subsection}{Put your R package on GitHub}

\url{https://jennybc.github.io/2014-05-12-ubc/ubc-r/session2.4_github.html}

\begin{itemize}
\item
  Change to the package directory
\item
  Initialize the repository with \texttt{git\ init}
\item
  Add and commit everything with

  \begin{enumerate}
  \def\labelenumi{\arabic{enumi}.}
  \tightlist
  \item
    \texttt{git\ add\ .} stage changes;
  \item
    \texttt{git\ status} optional check staged changes, but yet to submit;
  \item
    and \texttt{git\ commit} submit staged changes.
  \end{enumerate}
\item
  Create a \href{https://github.com/new}{new repository on GitHub}
\item
  Connect your local repository to the GitHub one

\begin{Shaded}
\begin{Highlighting}[]
\NormalTok{git remote add origin https://github.com/username/reponame}
\end{Highlighting}
\end{Shaded}
\item
  Push everything to github

\begin{Shaded}
\begin{Highlighting}[]
\NormalTok{git branch {-}M main}
\NormalTok{git push {-}u origin main}
\end{Highlighting}
\end{Shaded}
\end{itemize}

\chapter{Knit Rmd}\label{knit-rmd}

R Markdown is a powerful tool for combining analysis and reporting into the same document. R Markdown has grown substantially from a package that supports a few output formats, to an extensive and diverse ecosystem that supports the creation of books, blogs, scientific articles, websites, and even resumes.

Nice documentations

\begin{itemize}
\tightlist
\item
  \href{https://bookdown.org/yihui/rmarkdown}{R markdown: The definitive guide.} provides detailed references
\item
  \href{https://bookdown.org/yihui/rmarkdown-cookbook/}{R markdown cookbook} concise and covers essential functions, with examples.
\end{itemize}

\textbf{Quick takeaways}:

\begin{itemize}
\tightlist
\item
  Can still use horizontal separator ctrl + shift + S for dashed lines and ctrl + shift + = for equals
\item
  Headers must have one empty line above and below to separate it from other text
\end{itemize}

\textbf{YAML metadata}

Q: What is YAML?

A: YAML is a human-friendly data serialization language for all programming languages.

Q: What does YAML do?

A: It is placed at the very beginning of the document and is read by each of Pandoc, \textbf{rmarkdown}, and \textbf{knitr}.

\begin{itemize}
\tightlist
\item
  Provide metadata of the document.
\item
  located at the top of the file.
\item
  adheres to the YAML format and is delimited by lines containing three three dashes (\texttt{-\/-\/-}).
\end{itemize}

It can set values of the template variables, such as \texttt{title}, \texttt{author}, and \texttt{date} of the document.

\begin{itemize}
\item
  The \texttt{output} field is used by rmarkdown to apply the output format function \texttt{rmarkdown::html\_document()} in the rendering process.

  There are two types of output formats in the \textbf{rmarkdown} package: documents (e.g., \texttt{pdf\_document}), and presentations (e.g., \texttt{beamer\_presentation}).

  Supported output format examples: \texttt{html\_document}, \texttt{pdf\_document}.

  R Markdown documents (\texttt{html\_documents}) and R Notebook documents (\texttt{html\_notebook}) are very similar; in fact, an R Notebook document is a special type of R Markdown document. The main difference is using R Markdown document (\texttt{html\_documents}) you have to knit (render) the entire document each time you want to preview the document, even if you have made a minor change. However, using an R Notebook document (\texttt{html\_notebook}) you can view a preview of the final document without rendering the entire document.
\item
  Many aspects of the LaTeX template used to create PDF documents can be customized using \emph{top-level} \href{https://bookdown.org/yihui/rmarkdown/pdf-document.html\#tab:latex-vars}{YAML metadata} (note that these options do not appear underneath the \texttt{output} section, but rather appear at the top level along with \texttt{title}, \texttt{author}, and so on). For example:

\begin{Shaded}
\begin{Highlighting}[]
\SpecialCharTok{{-}{-}{-}}
\NormalTok{title}\SpecialCharTok{:} \StringTok{"Crop Analysis Q3 2013"}
\NormalTok{output}\SpecialCharTok{:}\NormalTok{ pdf\_document}
\NormalTok{fontsize}\SpecialCharTok{:} \DecValTok{11}\NormalTok{pt}
\NormalTok{geometry}\SpecialCharTok{:}\NormalTok{ margin}\OtherTok{=}\DecValTok{1}\DataTypeTok{i}\NormalTok{n}
\SpecialCharTok{{-}{-}{-}}
\end{Highlighting}
\end{Shaded}

  A few available metadata variables are displayed in the following (consult the Pandoc manual for \href{https://pandoc.org/MANUAL.html\#variables-for-latex}{the full list}):

  \begin{longtable}[]{@{}
    >{\raggedright\arraybackslash}p{(\linewidth - 2\tabcolsep) * \real{0.4340}}
    >{\raggedright\arraybackslash}p{(\linewidth - 2\tabcolsep) * \real{0.5660}}@{}}
  \toprule\noalign{}
  \begin{minipage}[b]{\linewidth}\raggedright
  Variable
  \end{minipage} & \begin{minipage}[b]{\linewidth}\raggedright
  Description
  \end{minipage} \\
  \midrule\noalign{}
  \endhead
  \bottomrule\noalign{}
  \endlastfoot
  \texttt{lang} & Document language code \\
  \texttt{fontsize} & Font size (e.g., \texttt{10pt}, \texttt{11pt}, or \texttt{12pt}) \\
  \texttt{documentclass} & LaTeX document class (e.g., \texttt{article}) \\
  \texttt{classoption} & Options for documentclass (e.g., \texttt{oneside}) \\
  \texttt{geometry} & Options for geometry class (e.g., \texttt{margin=1in}) \\
  \texttt{mainfont}, \texttt{sansfont}, \texttt{monofont}, \texttt{mathfont} & Document fonts (works only with \texttt{xelatex} and \texttt{lualatex}) \\
  \texttt{linkcolor}, \texttt{urlcolor}, \texttt{citecolor} & Color for internal links (cross references), external links (link to websites), and citation links (bibliography) \\
  \texttt{linestretch} & Options for line spacing (e.g.~1, 1.5, 3). \\
  \end{longtable}

  \begin{itemize}
  \item
    In PDFs, you can use code, typesetting commands (e.g., \texttt{\textbackslash{}vspace\{12pt\}}), and specific packages from LaTeX.

    \begin{enumerate}
    \def\labelenumi{\arabic{enumi}.}
    \tightlist
    \item
      The \texttt{header-includes} option loads LaTeX packages.
    \end{enumerate}

\begin{Shaded}
\begin{Highlighting}[]
\SpecialCharTok{{-}{-}{-}}
\NormalTok{output}\SpecialCharTok{:}\NormalTok{ pdf\_document}
\NormalTok{header}\SpecialCharTok{{-}}\NormalTok{includes}\SpecialCharTok{:}
\SpecialCharTok{{-}}\NormalTok{ \textbackslash{}usepackage\{fancyhdr\}}
\SpecialCharTok{{-}{-}{-}}

\NormalTok{\textbackslash{}pagestyle\{fancy\}}
\NormalTok{\textbackslash{}fancyhead[LE,RO]\{Holly Zaharchuk\}}
\NormalTok{\textbackslash{}fancyhead[LO,RE]\{PSY }\DecValTok{508}\NormalTok{\}}

\CommentTok{\# Problem Set 12}
\end{Highlighting}
\end{Shaded}

    \begin{enumerate}
    \def\labelenumi{\arabic{enumi}.}
    \setcounter{enumi}{1}
    \tightlist
    \item
      Alternatively, use \texttt{extra\_dependencies} to list a character vector of LaTeX packages. This is useful if you need to load multiple packages:
    \end{enumerate}

\begin{Shaded}
\begin{Highlighting}[]
\SpecialCharTok{{-}{-}{-}}
\NormalTok{title}\SpecialCharTok{:} \StringTok{"Untitled"}
\NormalTok{output}\SpecialCharTok{:} 
\NormalTok{  pdf\_document}\SpecialCharTok{:}
\NormalTok{    extra\_dependencies}\SpecialCharTok{:}\NormalTok{ [}\StringTok{"bbm"}\NormalTok{, }\StringTok{"threeparttable"}\NormalTok{]}
\SpecialCharTok{{-}{-}{-}}
\end{Highlighting}
\end{Shaded}

    f you need to specify options when loading the package, you can add a second-level to the list and provide the options as a list:

\begin{Shaded}
\begin{Highlighting}[]
\SpecialCharTok{{-}{-}{-}}
\NormalTok{title}\SpecialCharTok{:} \StringTok{"Untitled"}
\NormalTok{output}\SpecialCharTok{:} 
\NormalTok{  pdf\_document}\SpecialCharTok{:}
\NormalTok{    extra\_dependencies}\SpecialCharTok{:}
\NormalTok{      caption}\SpecialCharTok{:}\NormalTok{ [}\StringTok{"labelfont=\{bf\}"}\NormalTok{]}
\NormalTok{      hyperref}\SpecialCharTok{:}\NormalTok{ [}\StringTok{"unicode=true"}\NormalTok{, }\StringTok{"breaklinks=true"}\NormalTok{]}
\NormalTok{      lmodern}\SpecialCharTok{:}\NormalTok{ null}
\SpecialCharTok{{-}{-}{-}}
\end{Highlighting}
\end{Shaded}

    Here are some examples of LaTeX packages you could consider using within your report:

    \begin{itemize}
    \tightlist
    \item
      \href{https://ctan.org/pkg/pdfpages}{pdfpages}: Include full PDF pages from an external PDF document within your document.
    \item
      \href{https://ctan.org/pkg/caption}{caption}: Change the appearance of caption subtitles. For example, you can make the figure title italic or bold.
    \item
      \href{https://ctan.org/pkg/fancyhdr}{fancyhdr}: Change the style of running headers of all pages.
    \end{itemize}
  \item
    Some options are passed to Pandoc, such as \texttt{toc}, \texttt{toc\_depth}, and \texttt{number\_sections}. You should consult the \href{https://pandoc.org/MANUAL.html\#variables}{Pandoc documentation} when in doubt.

\begin{Shaded}
\begin{Highlighting}[]
\SpecialCharTok{{-}{-}{-}}
\NormalTok{output}\SpecialCharTok{:}
\NormalTok{  pdf\_document}\SpecialCharTok{:}
\NormalTok{    toc}\SpecialCharTok{:}\NormalTok{ true}
\NormalTok{        keep\_tex}\SpecialCharTok{:}\NormalTok{ true}
\SpecialCharTok{{-}{-}{-}}
\end{Highlighting}
\end{Shaded}

    \begin{itemize}
    \tightlist
    \item
      \texttt{keep\_tex:\ true} if you want to keep intermediate TeX. Easy to debug. Defaults to \texttt{false}.
    \end{itemize}
  \end{itemize}
\end{itemize}

We can include variables and R expressions in this header that can be referenced throughout our R Markdown document. For example, the following header defines \texttt{start\_date} and \texttt{end\_date} parameters, which will be reflected in a list called \texttt{params} later in the R Markdown document.

Thus, if we want to use these values in our R code, we can access them via \texttt{params\$start\_date} and \texttt{params\$end\_date}.

Should I use quotes to surround the values?

\begin{itemize}
\tightlist
\item
  Whenever applicable use the unquoted style since it is the most readable.
\item
  Use quotes when the value can be misinterpreted as a data type or the value contains a \texttt{:}.
\end{itemize}

\begin{Shaded}
\begin{Highlighting}[]
\CommentTok{\# values need quotes}
\NormalTok{foo}\SpecialCharTok{:} \StringTok{\textquotesingle{}\{\{ bar \}\}\textquotesingle{}} \CommentTok{\# need quotes to avoid interpreting as \textasciigrave{}dict\textasciigrave{} object}
\NormalTok{foo}\SpecialCharTok{:} \StringTok{\textquotesingle{}123\textquotesingle{}}       \CommentTok{\# need quote to avoid interpreting as \textasciigrave{}int\textasciigrave{} object}
\NormalTok{foo}\SpecialCharTok{:} \StringTok{\textquotesingle{}yes\textquotesingle{}}           \CommentTok{\# avoid interpreting as \textasciigrave{}boolean\textasciigrave{} object}
\NormalTok{foo}\SpecialCharTok{:} \StringTok{"bar:baz:bam"} \CommentTok{\# has colon, can be misinterpreted as key}

\CommentTok{\# values need not quotes}
\NormalTok{foo}\SpecialCharTok{:}\NormalTok{ bar1baz234}
\NormalTok{bar}\SpecialCharTok{:} \DecValTok{123}\NormalTok{baz}
\end{Highlighting}
\end{Shaded}

\section{Chunk Options}\label{chunk-options}

If you want to set chunk options globally, call \texttt{knitr::opts\_chunk\$set()} in a code chunk (usually the first one in the document), e.g.,

\begin{Shaded}
\begin{Highlighting}[]
\InformationTok{\textasciigrave{}\textasciigrave{}\textasciigrave{}\{r, label="setup", include=FALSE\}}
\InformationTok{knitr::opts\_chunk$set(}
\InformationTok{  comment = "\#\textgreater{}", echo = FALSE, fig.width = 6}
\InformationTok{)}
\InformationTok{\textasciigrave{}\textasciigrave{}\textasciigrave{}}
\end{Highlighting}
\end{Shaded}

Full list of chunk options: \url{https://yihui.org/knitr/options/}

Chunk options can customize nearly all components of code chunks, such as the source code, text output, plots, and the language of the chunk.

\textbf{Other languages are supproted in \texttt{Rmd}}

You can list the names of all available engines via:

\begin{Shaded}
\begin{Highlighting}[]
\FunctionTok{names}\NormalTok{(knitr}\SpecialCharTok{::}\NormalTok{knit\_engines}\SpecialCharTok{$}\FunctionTok{get}\NormalTok{())}
\DocumentationTok{\#\#  [1] "awk"          "bash"         "coffee"      }
\DocumentationTok{\#\#  [4] "gawk"         "groovy"       "haskell"     }
\DocumentationTok{\#\#  [7] "lein"         "mysql"        "node"        }
\DocumentationTok{\#\# [10] "octave"       "perl"         "php"         }
\DocumentationTok{\#\# [13] "psql"         "Rscript"      "ruby"        }
\DocumentationTok{\#\# [16] "sas"          "scala"        "sed"         }
\DocumentationTok{\#\# [19] "sh"           "stata"        "zsh"         }
\DocumentationTok{\#\# [22] "asis"         "asy"          "block"       }
\DocumentationTok{\#\# [25] "block2"       "bslib"        "c"           }
\DocumentationTok{\#\# [28] "cat"          "cc"           "comment"     }
\DocumentationTok{\#\# [31] "css"          "ditaa"        "dot"         }
\DocumentationTok{\#\# [34] "embed"        "eviews"       "exec"        }
\DocumentationTok{\#\# [37] "fortran"      "fortran95"    "go"          }
\DocumentationTok{\#\# [40] "highlight"    "js"           "julia"       }
\DocumentationTok{\#\# [43] "python"       "R"            "Rcpp"        }
\DocumentationTok{\#\# [46] "sass"         "scss"         "sql"         }
\DocumentationTok{\#\# [49] "stan"         "targets"      "tikz"        }
\DocumentationTok{\#\# [52] "verbatim"     "theorem"      "lemma"       }
\DocumentationTok{\#\# [55] "corollary"    "proposition"  "conjecture"  }
\DocumentationTok{\#\# [58] "definition"   "example"      "exercise"    }
\DocumentationTok{\#\# [61] "hypothesis"   "proof"        "remark"      }
\DocumentationTok{\#\# [64] "solution"     "marginfigure"}
\end{Highlighting}
\end{Shaded}

The engines from \texttt{theorem} to \texttt{solution} are only available when you use the \textbf{bookdown} package, and the rest are shipped with the \textbf{knitr} package.

To use a different language engine, you can change the language name in the chunk header from \texttt{r} to the engine name, e.g.,

\begin{Shaded}
\begin{Highlighting}[]
\StringTok{\textasciigrave{}\textasciigrave{}\textasciigrave{}}\AttributeTok{python}
\AttributeTok{x = \textquotesingle{}hello, python world!\textquotesingle{}}
\AttributeTok{print(x.split(\textquotesingle{} \textquotesingle{}))}
\StringTok{\textasciigrave{}\textasciigrave{}\textasciigrave{}}
\end{Highlighting}
\end{Shaded}

For engines that rely on external interpreters such as \texttt{python}, \texttt{perl}, and \texttt{ruby}, the default interpreters are obtained from \texttt{Sys.which()}, i.e., using the interpreter found via the environment variable \texttt{PATH} of the system. If you want to use an alternative interpreter, you may specify its path in the chunk option \texttt{engine.path}.

For example, you may want to use Python 3 instead of the default Python 2, and we assume Python 3 is at \texttt{/usr/bin/python3}

\begin{Shaded}
\begin{Highlighting}[]
\InformationTok{\textasciigrave{}\textasciigrave{}\textasciigrave{}\{python, engine.path = \textquotesingle{}/usr/bin/python3\textquotesingle{}\}}
\InformationTok{import sys}
\InformationTok{print(sys.version)}
\InformationTok{\textasciigrave{}\textasciigrave{}\textasciigrave{}}
\end{Highlighting}
\end{Shaded}

\begin{itemize}
\tightlist
\item
  All outputs support markdown syntax.
\item
  If the output is html, you can write in html syntax.
\end{itemize}

The \textbf{chunk label} for each chunk is assumed to be unique within the document. This is especially important for cache and plot filenames, because these filenames are based on chunk labels. Chunks without labels will be assigned labels like \texttt{unnamed-chunk-i}, where \texttt{i} is an incremental number.

\begin{itemize}
\item
  Chunk label doesn't need a \texttt{tag}, i.e., you only provide the \texttt{value}.
\item
  If you prefer the form \texttt{tag=value}, you could also use the chunk option \texttt{label} explicitly, e.g.,

\begin{Shaded}
\begin{Highlighting}[]
\InformationTok{\textasciigrave{}\textasciigrave{}\textasciigrave{}\{r, label=\textquotesingle{}my{-}chunk\textquotesingle{}\}}
\InformationTok{\# one code chunk example}
\InformationTok{\textasciigrave{}\textasciigrave{}\textasciigrave{}}
\end{Highlighting}
\end{Shaded}
\end{itemize}

You may use \texttt{knitr::opts\_chunk\$set()} to change the default values of chunk options in a document.

\textbf{Commonly used chunk options}

\begin{itemize}
\tightlist
\item
  Complete list \href{https://yihui.org/knitr/options/}{here}. Or \texttt{?opts\_chunk} to get the help page.
\end{itemize}

\begin{longtable}[]{@{}
  >{\raggedright\arraybackslash}p{(\linewidth - 2\tabcolsep) * \real{0.2500}}
  >{\raggedright\arraybackslash}p{(\linewidth - 2\tabcolsep) * \real{0.7500}}@{}}
\toprule\noalign{}
\begin{minipage}[b]{\linewidth}\raggedright
Options
\end{minipage} & \begin{minipage}[b]{\linewidth}\raggedright
Definitions
\end{minipage} \\
\midrule\noalign{}
\endhead
\bottomrule\noalign{}
\endlastfoot
\texttt{echo=TRUE} & Whether to display the \textbf{source code} in the output document.Use this when you want to show the output but not the code itself. \\
\texttt{eval=TRUE} & Whether to evaluate the code chunk. \\
\texttt{include=TRUE} & Whether to include the {chunk \textbf{output}} in the output document. If \texttt{FALSE}, nothing will be written into the output document, but the code is still evaluated and plot files are generated if there are any plots in the chunk, so you can manually insert figures later. \\
\texttt{message=TRUE} & Whether to preserve messages emitted by \texttt{message()} \\
\texttt{warning=TRUE} & Whether to show warnings in the output produced by \texttt{warning()}. \\
\texttt{results=\textquotesingle{}markup\textquotesingle{}} & Controls how to display the text results. When \texttt{results=\textquotesingle{}markup\textquotesingle{}} that is to write text output as-is, i.e., write the raw text results directly into the output document without any markups.Useful when priting \texttt{stargazer} tables. \\
\texttt{comment=\textquotesingle{}\#\#\textquotesingle{}} & The prefix to be added before each line of the text output. Set \texttt{comment\ =\ \textquotesingle{}\textquotesingle{}} remove the default \texttt{\#\#}. \\
\texttt{fig.keep=\textquotesingle{}high\textquotesingle{}} & How plots in chunks should be kept. \texttt{high}: Only keep high-level plots (merge low-level changes into high-level plots). \texttt{none}: Discard all plots. \texttt{all}: Keep all plots (low-level plot changes may produce new plots). \texttt{first}: Only keep the first plot. \texttt{last}: Only keep the last plot. If set to a numeric vector, the values are indices of (low-level) plots to keep.If you want to choose the second to the fourth plots, you could use \texttt{fig.keep\ =\ 2:4} (or remove the first plot via \texttt{fig.keep\ =\ -1}). \\
\texttt{fig.align="center"} & Figure alignment. \\
\texttt{fig.pos="H"} & A character string for the figure position arrangement to be used in \texttt{\textbackslash{}begin\{figure\}{[}{]}}. \\
\texttt{fig.cap} & Figure caption. \\
\end{longtable}

{\texttt{results=\textquotesingle{}markup\textquotesingle{}}} note plural form for result\textbf{s}.

\begin{itemize}
\item
  \texttt{markup}: Default. Mark up text output with the appropriate environments depending on the output format. For example, for R Markdown, if the text output is a character string \texttt{"{[}1{]}\ 1\ 2\ 3"}, the actual output that \textbf{knitr} produces will be:

\begin{Shaded}
\begin{Highlighting}[]
\StringTok{\textasciigrave{}\textasciigrave{}\textasciigrave{}}
\AttributeTok{[1] 1 2 3}
\StringTok{\textasciigrave{}\textasciigrave{}\textasciigrave{}}
\end{Highlighting}
\end{Shaded}

  In this case, \texttt{results=\textquotesingle{}markup\textquotesingle{}} means to put the text output in fenced code blocks (```).
\item
  \texttt{asis}: Write text output as-is, i.e., write the raw text results directly into the output document without any markups.

\begin{Shaded}
\begin{Highlighting}[]
\InformationTok{\textasciigrave{}\textasciigrave{}\textasciigrave{}\{r, results=\textquotesingle{}asis\textquotesingle{}\}}
\InformationTok{cat("I\textquotesingle{}m raw **Markdown** content.\textbackslash{}n")}
\InformationTok{\textasciigrave{}\textasciigrave{}\textasciigrave{}}
\end{Highlighting}
\end{Shaded}

  Sometime, you encounter the following error messages when you have R codes within \texttt{enumerate} environment.

  \begin{quote}
  You can't use `macro parameter character \#' in horizontal mode.
  \end{quote}

  By default, knitr prefixes R output with \texttt{\#\#}, which can't be present in your TeX file.

  Solution:

  \begin{itemize}
  \tightlist
  \item
    specify \texttt{results="asis"} in code chunks.
  \end{itemize}
\item
  \texttt{hold}: Hold all pieces of text output in a chunk and flush them to the end of the chunk.
\item
  \texttt{hide} (or \texttt{FALSE}): Hide text output.
\end{itemize}

\begin{center}\rule{0.5\linewidth}{0.5pt}\end{center}

\section{Print Verbatim R code chunks}\label{print-verbatim-r-code-chunks}

\textbf{Including verbatim R code chunks inside R Markdown}

One solution for including verbatim R code chunks (see below for more) is to insert hidden inline R code (\texttt{\textasciigrave{}r\ \ \ \textquotesingle{}\textquotesingle{}\textasciigrave{}}) immediately before or after your R code chunk.

\begin{itemize}
\tightlist
\item
  The hidden inline R code will be evaluated as an inline expression to an empty string by knitr.
\end{itemize}

Then wrap the whole block within a markdown code block. The rendered output will display the verbatim R code chunk --- including backticks.

R code generating the four backticks block:

\begin{Shaded}
\begin{Highlighting}[]
\NormalTok{output\_code }\OtherTok{\textless{}{-}}
\StringTok{"\textasciigrave{}\textasciigrave{}\textasciigrave{}\textasciigrave{}markdown}
\StringTok{\textasciigrave{}\textasciigrave{}\textasciigrave{}\{r\}}
\StringTok{plot(cars)}
\StringTok{\textasciigrave{}\textasciigrave{}\textasciigrave{} }\SpecialCharTok{\textbackslash{}n}\StringTok{\textasciigrave{}\textasciigrave{}\textasciigrave{}\textasciigrave{}"}
\FunctionTok{cat}\NormalTok{(output\_code)}
\end{Highlighting}
\end{Shaded}

Write this code in your R Markdown document:

\begin{verbatim}
````markdown
`r ''````{r}
plot(cars)
``` 
````
\end{verbatim}

or

\begin{verbatim}
````markdown
```{r}`r ''`
plot(cars)
``` 
````
\end{verbatim}

Knit the document and the code will render like this in your output:

\begin{Shaded}
\begin{Highlighting}[]
\InformationTok{\textasciigrave{}\textasciigrave{}\textasciigrave{}\{r\}}
\InformationTok{plot(cars)}
\InformationTok{\textasciigrave{}\textasciigrave{}\textasciigrave{}}
\end{Highlighting}
\end{Shaded}

\textbf{References}:

\url{https://yihui.org/en/2017/11/knitr-verbatim-code-chunk/}

\url{https://support.posit.co/hc/en-us/articles/360018181633-Including-verbatim-R-code-chunks-inside-R-Markdown}

\url{https://themockup.blog/posts/2021-08-27-displaying-verbatim-code-chunks-in-xaringan-presentations/}

  \bibliography{book.bib,packages.bib}

\end{document}
